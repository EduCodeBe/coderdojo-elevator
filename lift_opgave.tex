\documentclass[a4paper, 10pt]{article}

\usepackage[a4paper,left=3cm,right=2cm,top=2.5cm,bottom=2.5cm]{geometry}


\usepackage{hyperref}
\usepackage{graphicx}

% defaults voor code
\usepackage{listings}
\usepackage{color}
\definecolor{mygray}{rgb}{0.5,0.5,0.5}
\lstset{
    numbers=left,
    numberstyle=\small\color{mygray},
    keywordstyle=\color{blue},
    stringstyle=\color{green},
    numbersep=10pt,
}

% Een deftig font dat geen zeer doet aan de oogjes
\usepackage{fontspec}
\setmainfont{Helvetica Neue Light}[
  ItalicFont=Helvetica Neue Light Italic,
]


\usepackage[parfill]{parskip}

% We spreken Nederlands.
\usepackage[dutch]{babel}


\begin{document}

\title{We bouwen een lift met Dwengo}
\author{CoderDojo's Bredene - Oostende}
\maketitle

\section{Introductie}

We zullen in dit project een lift bouwen die we kunnen aansturen met behulp van een elektronica-bordje, in dit
geval een {\em dwenguino}. Hiervoor hebben we de volgende componenten nodig:

\begin{itemize}
\item een dwenguino,
\item een computer met daarop de Arduino IDE software,
\item de dwengo-bibliotheken voor de Arduino IDE\footnote{\url{http://www.dwengo.org/install}}
\item een liftkoker,
\item een lift, die bij voorkeur niet kan ronddraaien terwijl ze verplaatst wordt,
\item een motor die kan aangestuurd worden door de dwenguino
\item kabels om de motor van stroom te voorzien
\end{itemize}

De liftkoker en de lift zijn reeds gebouwd voor gebruik in onze Dojo, maar als je zelf een koker wil bouwen, waarop
we een van onze motoren kunnen plaatsen, dan mag dat natuurlijk ook.

We hebben dit project opgesplitst in een aantal deeltaken, die normaal van makkelijk of vrij eenvoudig naar redelijk
moeilijk en vrij complex evolueren. Op dit moment zijn nog niet al die taken volledig uitgewerkt :)


\section{Aankoppelen van de dweguino aan de motor en de lift verplaatsen}

In deze eerste taak zullen we ervoor zorgen dat we de eenvoudigweg op en neer kunnen verplaatsen over een zelfde
afstand. Hiervoor hebben we een {\em sketch} nodig die we schrijven in een gewone tekstgebaseerde programmeertaal (dus
niet in Scratch). We kunnen deze sketch schrijven in o.a. de Arduino IDE.

Als je de IDE opent zie je onmiddellijk de standaard sketch:

\begin{lstlisting}[language=C++]
void setup() {
  // put your setup code here, to run once:

}

void loop() {
  // put your main code here, to run repeatedly:

}
\end{lstlisting}

De code die nodig is om alles in te stellen, plaatsen we typisch in \texttt{setup()}, de code die telkens weer moet worden
uitgevoerd om bijvoorbeeld te reageren op gebeurtenissen zoals het indrukken van een knop, een sensor die iets opmerkt,
\ldots, plaatsen we in \texttt{loop()}.

Variabelen die we in meerdere functies zullen gebruiken, plaatsen we buiten elke functie en zijn dan zogenaamde globale
variabelen. Je kunt dit een beetje vergelijken met een variabele in Scratch die bruikbaar is door alle sprites.

In onze eerste taak hebben we zo'n variabele nodig om de motor te kunnen gebruiken. Omdat de computer moet weten wat we
bedoelen met onze variabele, geven we ze ook een {\em type}. Dit bepaalt o.a. wat we verder kunnen aanvangen met onze
variabele. In ons geval heeft de variabele het type \texttt{DCMotor}. Opdat de computer zou weten wat dit precies is,
moeten we in ons programma ook vertellen waar de definitie van die \texttt{DCMotor} staat. Dit doen we door stukjes code
in te laden in onze broncode. Om te vermijden dat we alles meermaals moeten typen bestaat hier een handig statement voor
in C of C++, namelijk:

\begin{lstlisting}
#include <Dwenguino.h>
#include <DwenguinoMotor.h> // the motor library
\end{lstlisting}

Typisch plaatsen we deze statements bovenaan ons programma zodat alle code die volgt tevens de definities kan gebruiken
die we hier in ons programma hebben geladen.

Jouw eerste opdracht is nu de voorbeelden in de Arduino IDE te bekijken en na te gaan op welke manier we een
\texttt{DCMotor} kunnen defini\"eren en aansturen. Zorg er vooral voor dat je probeert de code te begrijpen en ze
niet gewoon overtypt of kopieert.

Denk ook eens na over de volgende vragen:

\begin{enumerate}
\item hoe snel kan de motor draaien?
\item hoe laat je de motor in de andere richting draaien?
\end{enumerate}

Probeer de motor een tijdje in de ene richting en een tijdje in de andere richting te laten draaien. Je zult zien dat
de motor niet onmiddellijk van richting verandert, maar dat ze eerst stopt. Dit is ingebouwd in de Dwengo bibliotheek
die we gebruiken in ons programma om te vermijden dat de motor snel stuk gaat. {\em Zorg er in deze fase voor dat de motor de
lift nog niet verplaatst door het tandwiel van de katrolas een beetje te verplaatsen.}

\section{De lift bewegen met twee drukknoppen op de dwenguino}

Nu we weten hoe we de motor kunnen bewegen in beide richtingen, is het tijd om de volgende stap te zetten. We zullen
de lift laten bewegen door op een knop te drukken die aangeeft dat we omhoog willen en door op een knop te drukken
die aangeeft dat we de lift omlaag willen verplaatsen.

We hebben hiervoor enkele nieuwe functies en constanten nodig:

\begin{itemize}
\item \texttt{digitalRead()} leest de waarde uit van een drukknop,
\item \texttt{delay()} pauzeert de uitvoering van het programma gedurende het opgegeven aantal milliseconden,
\item de constanten die de knoppen defini\"eren: \texttt{SW\_E} (oost), \texttt{SW\_W} (west), \texttt{SW\_S} (zuid), \texttt{SW\_N} (noord) en \texttt{SW\_C} (centrum),
\item de constante die aangeeft dat een knop is ingedrukt: \texttt{PRESSED},
\item een statement dat overeenkomt met het {\em Als ... Dan} blok uit Scratch: \texttt{if (voorwaarde) \{ /* doe iets */\}}.
\end{itemize}

Definieer ook zelf een functie om de lift een stukje omhoog te laten gaan en een gelijkaardige functie om ze (evenveel)
omlaag te laten gaan. Zorg er vervolgens voor dat je deze functie oproept op het ogenblik dat er op de knop wordt
gedrukt.

\textbf{Let op!} Om te vermijden dat het mogelijk is te snel te wisselen van richting door de knoppen afwisselend in te
drukken, is het belangrijk dat de \texttt{loop()}-functie telkens wacht nadat een knop is ingedrukt. Daarvoor gebruik je
de \texttt{delay()}-functie.

\section{Bepalen van de beweging van de lift}

Voor we de lift per verdieping kunnen verplaatsen, moeten we eerst achterhalen hoe ver de lift zich verplaatst voor
een gegeven motorsnelheid en de tijd die we de motor laten draaien. Er zijn verschillende manieren om dit te bepalen,
en in dit onderdeel zullen we experimenteel bepalen hoe ver de lift kan bewegen voor een aantal snelheden. We laten
de motor telkens 1s draaien. Let op! Zodra je merkt dat de verplaatsing te groot wordt en de lift dus te ver omhoog
komt, kun je de snelheid niet meer verhogen.

Om de gegevens netjes bij te houden open je best een spreadsheet-applicatie zoals Excel, Numbers (op de Mac), of Google
docs als je een netwerkverbinding hebt. We hebben twee kolommen nodig: eentje voor de snelheid en eentje voor de afgelegde
afstand.

Welke snelheden zou je kiezen als je zo snel mogelijk een idee wil hebben van het verband tussen de snelheid
en de verplaatsing?


\end{document}
